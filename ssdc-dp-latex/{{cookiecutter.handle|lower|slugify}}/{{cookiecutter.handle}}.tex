\documentclass[DP]{spherex}
% spherex class documentation: https://github.com/SPHEREx/spherex-tex/blob/main/README.md

\input{meta}

\spherexHandle{ {{- cookiecutter.handle -}} }
\title{ {{- cookiecutter.title -}} }

\version{0.1}
% \docDate{YYYY-MM-DD} % uncomment to override the vcs date (see meta.tex)
% \approved{YYYY-MM-DD}{Approver Name} % uncomment to set approval metadata

\ipaclead[email= {{- cookiecutter.ipac_lead_email -}} ]{ {{- cookiecutter.ipac_lead_name -}} }

% Additional authors, if any, separated by "\\":
%% \author{
%%  \person[email=galileo@example.com]{Galileo~Galilei} \\
%%  \person{Isaac~Newton}
%%}


\begin{document}
\maketitle

\begin{dochistory}
% Add later versions at the bottom.
\addtohist{0.1}{2020-01-01}{Initial draft.}{Galileo Galilei}
\end{dochistory}

\section{Overview}

% Briefly describe the nature of the data product.  Edit as needed:

This document defines the {{ cookiecutter.title -}}.
This data product is produced in the SSDC Pipelines at Level~XXX.
It ( is / is not ) one of the data products intended for public release.

\subsection{Scope}

% State what this document covers authoritatively and what it does not cover.
% For example (change if necessary):

This document defines only the data format of individual instances of this
data product.
It does not describe how they are produced.
It also does not describe how the full collection of these data products are
organized in storage, what external metadata tables are associated with them,
or how they are accessed by users.

This is a document at Level 4 in the SPHEREx systems engineering hierarchy
and pertains only to the SPHEREx Science Data Center project element.

\subsection{Nomenclature}

The SPHEREx Science Data Center (SSDC) is an organization located at
the California Institute of Technology / IPAC and is responsible for the
SPHEREx Level~1--3 science data processing and archiving.

% Add to or edit this list as necessary:

\begin{description}
  \item[ADCS]  Attitude Determination and Control System: a component of
    the SPHEREx spacecraft responsible for maintaining the orientation of
    the spacecraft in orbit and its pointing at designated targets.
  \item[FITS]  Flexible Image Transport System: an IAU standard for the
    representation of astronomical images in files.
  \item[IVOA]  International Virtual Observatory Alliance: an international
    forum for establishing standards for interoperability of astronomical
    data products and services.
  \item[LVF]   Linear Variable Filter: a type of optical filter in which the
    bandpass varies continuously across the filter.
  \item[SUR]   Sample-Up-the-Ramp: the readout method used for the SPHEREx
    detectors, in which the detector is read out at an approximately 1.5s
    cadence and the resulting pixel values are fitted with a slope
    representing the uncalibrated flux incident on the pixel.
\end{description}

\subsection{Change Control}

% Describe how the data product's change control flow works.
% Who has to approve a change?

This document is under change control at Level 4, within the SSDC.
Changes require approval by the SSDC task lead or a designee.

As part of the interface with the Science Team
and end users%%% remove for data products not released externally
, all substantive changes require a review by a representative of
the SPHEREx Science Team.

\subsection{Document Release Control}

% Modify as needed:

This document is ultimately intended to be suitable for public release along
with the SPHEREx data products, and should be edited accordingly.

However, this version of this document has not been formally reviewed from
export control or other document release perspectives.

\subsection{Applicable Documents}

% Identify key documents related to this data product

\begin{itemize}
  \item SPHEREx Science Data Center Level 4 Requirements, JPL D-104005
  \item SPHEREx Data Management Plan, JPL A4333896
  \item SPHEREx Science Data Center Operations Concept, Caltech/IPAC SSDC-PM-001
  \item SPHEREx Science Team --- SSDC Operational Interface Agreements, Caltech/IPAC SSDC-IF-002
 % Delete the next line for data products not released through IRSA:
  \item NASA IRSA --- SSDC Operational Interface Agreements, Caltech/IPAC SSDC-IF-003
\end{itemize}


\section{Production}

% Briefly describe where in the system the data product is created, e.g., in
% which pipeline module, but do not describe _how_ it is produced.


\section{Release}

% Briefly describe whether and how the data product is released outside the SSDC
% or whatever other entity creates it.


\section{Detailed Description}

% NB: This section may be repeated if the document describes multiple data products.

% Provide a full narrative description of the data product itself.
% Try not to repeat details of _how_ it is produced (e.g., from a pipeline module description).
% 
% If it's an image data product, describe the HDUs in the FITS file and the headers that are used.
% If it's a catalog product, provide a table with column names, data types, units, UCDs, descriptions, etc.


\subsection{Data Product Size}

% *Following* the details, we conclude with an estimate of the size of an
% individual instance of the data product, for images, or an estimate of
% the size of a catalog per row.
%
% Delete for DP documents that are about common internal details or otherwise
% not about a complete data product.



% spherex.bib is distributed with spherex-tex and maintained at
% https://github.com/SPHEREx/spherex-tex/blob/main/texmf/bibtex/bib/spherex.bib
\bibliography{spherex}

\end{document}
